\chapter{Kết luận và Hướng phát triển}
\label{Chapter6}

\emph{Chương này trình bày kết luận sau quá trình thực hiện đề tài, bao gồm kỹ năng nghiên cứu, kỹ năng xây dựng một ứng dụng hoàn chỉnh, kết quả tóm tắt, ý nghĩa và khuyết điểm của đề tài. Cuối cùng là hướng phát triển và những định hướng trong tương lai.}

\section{Kết luận}
\label{sec:ket-luan}

\subsection{Tóm tắt kết quả đạt được}

Trong quá trình thực hiện khóa luận, nhóm em đã học hỏi và đạt được một số kết quả. Cụ thể:

\begin{itemize}
    \item[--] Về mặt ứng dụng
        \begin{itemize}
            \item[\textbullet] Xây dựng thành công ứng dụng chatbot chỉ đường với dữ liệu đầu vào và đầu ra là giọng nói và văn bản.
        \end{itemize}
    \item[--] Về mặt kĩ năng mềm
        \begin{itemize}
            \item[\textbullet] Kỹ năng đọc tài liệu và phân tích vấn đề
            \item[\textbullet] Kĩ năng làm việc nhóm và phân chia công việc
            \item[\textbullet] Kĩ năng trình bày và thuyết trình, viết báo cáo
        \end{itemize}
\end{itemize}

\subsection{Ý nghĩa}
% Bên cạnh những kết quả, việc hoàn thành đề tài Xây dựng giải pháp trả lời tự động (chatbot) bằng tiếng Việt cũng mang lại những ý nghĩa đáng kể:

% \begin{itemize}
%     \item[--] Kết quả của đề tài là minh chứng cho tính khả dụng và khả thi của đề tài
%     \item[--] Đề tài cũng chứng minh ý nghĩa ....
% \end{itemize}

\subsection{Những hạn chế, giới hạn}
Ứng dụng hoạt động tương đối tốt. Tuy nhiên, vẫn còn nhiều mặt hạn chế nhất định như sau:
\begin{itemize}
    \item[--] Đối với vấn đền phân tích intent và slot đôi khi vẫn còn nhầm lẫn và chưa được chính xác.
    \item[--] Đối với từ điển dịch tự tạo còn ít từ và nhiều hạn chế khi dịch.
\end{itemize}

\section{Hướng phát triển}