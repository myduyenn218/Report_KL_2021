\chapter{Kết quả}
\label{Chapter6}

\emph{Chương này trình bày kết quả đề tài, kết quả phát hiện intent, trích xuất thông tin và các chức năng đã cài đặt được.}


Xây dựng thành công giải pháp trả lời tự động bằng tiếng Việt với các chức năng chi tiết như sau:.
\begin{itemize}
    \item[--] Xây dựng thành công ứng dụng chatbot trên hai nền tảng Android và iOS.
    \item[--] Chatbot sử dụng ngôn ngữ trò chuyện chính là Tiếng Việt.
    \item[--] Ứng dụng cho phép người dùng tương tác với chatbot bằng giọng nói và văn bản.
\end{itemize}

Tự tạo một bộ dữ liệu huấn luyện bao gồm 80 câu huấn luyện và 40 câu kiểm thử cho 4 loại ý định (intent) trong lĩnh vực tìm đường đi. Bộ dữ liệu được tạo ban đầu với ngôn ngữ tiếng Việt và sử dụng các phương pháp chuyển đổi ngôn ngữ và bộ dịch và sau cùng tạo được một bản dữ liệu tiếng Anh được chuyển đổi tự động từ bản tiếng Việt.  

Sử dụng thư viện mã nguồn mở Snips NLU\cite{Snipsnlu} để xây dựng mô hình phát hiện ý định (intent) và trích xuất thông tin từ văn bản. Sử dụng Google Map API\cite{ggmaps} để tìm kiếm câu trả lời phù hợp và trả về cho người dùng. Dưới đây là chi tiết của kết quả huấn luyện dữ liệu và ứng dụng.

\section{Kết quả huấn luyện dữ liệu}

Có rất nhiều cách đánh giá một mô hình phân lớp. Tuỳ vào những bài toán khác nhau mà chúng ta sử dụng các phương pháp khác nhau. Các phương pháp thường được sử dụng là: Độ chính xác (accuracy score), confusion matrix, ROC curve, Area Under the Curve, Precision and Recall, F1 score, Top R error,... Chúng em sử dụng đánh giá F1 Score để đánh giá phân lớp các mô hình. Sau đây là khái quát một số khái niệm để chúng ta hiểu rõ hơn về kết quả đánh giá.

\begin{itemize}
    \item[--] Độ chính xác (Accuracy) là tỉ lệ giữa số điểm được phân loại đúng và tổng số điểm trong dữ liệu kiểm thử. Accuracy chỉ phù hợp với các bài toán mà kích thước các lớp dữ liệu là tương đối như nhau.
    \item[--] Confusion matrix giúp có cái nhìn rõ hơn về việc các điểm dữ liệu được phân loại đúng/sai như thế nào.
    \item[--] True Positive (TP): số lượng điểm của lớp positive được phân loại đúng là positive.
    \item[--] True Negative (TN): số lượng điểm của lớp negative được phân loại đúng là negative.
    \item[--] False Positive (FP): số lượng điểm của lớp negative bị phân loại nhầm thành positive.
    \item[--] False Negative (FN): số lượng điểm của lớp positive bị phân loại nhầm thành negative
\end{itemize}

Precision được định nghĩa là tỉ lệ số điểm true positive trong số những điểm được phân loại là positive (TP + FP).
\begin{center}
    Precision = $\dfrac{TP}{TP+EP}$
\end{center}

Recall được định nghĩa là tỉ lệ số điểm true positive trong số những điểm thực sự là positive (TP + FN).
\begin{center}
    Recall = $\dfrac{TP}{TP+FN}$
\end{center}

$F_{1}$-score có giá trị nằm trong nửa khoảng (0,1]. F1 càng cao, bộ phân lớp càng tốt. Khi cả recall và precision đều bằng 1 (tốt nhất có thể), F1=1
\begin{center}
    $F_{1}=2\frac{1}{\frac{1}{\text{Precision}} + \frac{1}{\text{recall}}}= 2\frac{\text{precision} \times \text{recall}}{\text{precision} + \text{recall}}$
\end{center}

Chúng em xây dựng mô hình phân loại các câu hỏi chỉ đường. Nhãn dán của của các quan sát sẽ bao gồm: câu hỏi chỉ đường (findRouteAB), câu hỏi khoảng cách (askDistance), câu hỏi vị trí hiện tại (askLocation), câu hỏi địa chỉ của một địa danh (askAddress). Kích thước của tập dữ liệu như sau:
\begin{itemize}
    \item[--] Tập train: 80 câu bao gồm mỗi loại 20 câu.
    \item[--] Tập test: 40 câu bao gồm mỗi loại 10 câu.
\end{itemize}

Sau khi lựa chọn các loại intent để thực hiện và đã tạo dự liệu để huấn luyện và tiến hành kiểm thử cho từng intent.

Đem bộ dữ liệu để huấn luyện và đánh giá, ta đạt được Confusion matrix như hình \ref{fig:metrics-dict-end}:


\begin{table}[]
\begin{center}
\scalebox{0.8}{
\begin{tabular}{|l|l|llllll}
\hline
\textbf{Intent Name} & \multicolumn{3}{l|}{\textbf{Intent Classification}}                                          & \multicolumn{4}{l|}{\textbf{Slot Filling}}                                                                                                                                                                                                                                                         \\ \hline
\textbf{}            & \textbf{Precision} & \multicolumn{1}{l|}{\textbf{Recall}} & \multicolumn{1}{l|}{\textbf{F1}} & \multicolumn{1}{l|}{\textbf{Slot}}                                                  & \multicolumn{1}{l|}{\textbf{Precision}}                            & \multicolumn{1}{l|}{\textbf{Recall}}                               & \multicolumn{1}{l|}{\textbf{F1}}                                   \\ \hline
askAddress           & 0.769              & \multicolumn{1}{l|}{1}               & \multicolumn{1}{l|}{0.870}       & \multicolumn{1}{l|}{location}                                                       & \multicolumn{1}{l|}{1}                                             & \multicolumn{1}{l|}{1}                                             & \multicolumn{1}{l|}{1}                                             \\ \hline
askDistance          & 1                  & \multicolumn{1}{l|}{1}               & \multicolumn{1}{l|}{1}           & \multicolumn{1}{l|}{\begin{tabular}[c]{@{}l@{}}location\\ desLocation\end{tabular}} & \multicolumn{1}{l|}{\begin{tabular}[c]{@{}l@{}}1\\ 1\end{tabular}} & \multicolumn{1}{l|}{\begin{tabular}[c]{@{}l@{}}1\\ 1\end{tabular}} & \multicolumn{1}{l|}{\begin{tabular}[c]{@{}l@{}}1\\ 1\end{tabular}} \\ \hline
askLocation          & 1                  & \multicolumn{1}{l|}{0.7}             & \multicolumn{1}{l|}{0.824}       & \multicolumn{1}{l|}{}                                                               & \multicolumn{1}{l|}{}                                              & \multicolumn{1}{l|}{}                                              & \multicolumn{1}{l|}{}                                              \\ \hline
findRouteAB          & 1                  & \multicolumn{1}{l|}{1}               & \multicolumn{1}{l|}{1}           & \multicolumn{1}{l|}{\begin{tabular}[c]{@{}l@{}}location\\ desLocation\end{tabular}} & \multicolumn{1}{l|}{\begin{tabular}[c]{@{}l@{}}1\\ 1\end{tabular}} & \multicolumn{1}{l|}{\begin{tabular}[c]{@{}l@{}}1\\ 1\end{tabular}} & \multicolumn{1}{l|}{\begin{tabular}[c]{@{}l@{}}1\\ 1\end{tabular}} \\ \hline
Average metrics      & 0.942              & \multicolumn{1}{l|}{0.925}           & \multicolumn{1}{l|}{0.923}       & \multicolumn{1}{l|}{}                                                               & \multicolumn{1}{l|}{1}                                             & \multicolumn{1}{l|}{1}                                             & \multicolumn{1}{l|}{1}                                             \\ \hline
Accuracy             & 0.925              &                                      &                                  &                                                                                     &                                                                    &                                                                    &                                                                    \\ \cline{1-2}
\end{tabular}}
 \caption{Các chỉ số của mô hình}
    \label{fig:metrics-dict-end}

\end{center}

\end{table}

Các chỉ số được tính cho từng lớp riêng lẻ, ngoại trừ accuracy (độ chính xác); nó được tính toán trên tất cả các lớp để cung cấp độ chính xác phân loại cho mô hình này.

Đánh giá tổng quan: Accuracy (Độ chính xác của toàn bộ dữ liệu) là 0.925.

Đánh giá chi tiết từng phân loại: Dựa trên kết quả F1-Score cho từng phân loại, dễ dàng nhìn thấy độ chính xác của từng loại sẽ được sắp xếp theo thứ tự sau đây: findRouteAB, askDistance, askLocation, askAddress.

Như vậy, dựa trên kết quả kiểm thử cho thấy độ chính xác khi phát hiện các ý định (intent) và các thực thể (entity) là cao, điều này giúp cho việc phân loại câu hỏi chính xác giúp ứng dụng có thể đưa ra những câu trả lời phù hợp với từng nhu cầu của người dùng.

Dưới đây là kết quả ví dụ với câu đầu vào: "Đường đi từ Đại học Nông lâm Thành phố Hồ Chí Minh đến Đại học Kinh tế Luật" (xem hình \ref{fig:detect-intent}).

\begin{figure}[htp]
    \centering
    \includegraphics[width=15cm]{images/detect_intent.jpg}
    \caption{Các chỉ số của mô hình}
    \label{fig:detect-intent}
\end{figure}

Với câu đầu vào như trên, hệ thống sẽ chuyển hoá từ ngữ từ tiếng Việt sang tiếng Anh (ngoại trừ các địa danh), sau đó hệ thống sẽ tiến hành dự đoán ý định (intent) và trích xuất các thực thể (entity) thu được kết quả như hình \ref{fig:detect-intent}.
\begin{itemize}
    \item[--] Phân loại intent: "findRouteAB"
    \item[--] Địa điểm bắt đầu: "Đại học Nông lâm Thành phố Hồ Chí Minh"
    \item[--] Địa điểm kết thúc: "Đại học Kinh tế Luật"
\end{itemize}

\section{Ứng dụng}
\subsection{Nền tảng phát triển}
Hiện tại, ứng dụng đã được phát triển trên hai nền tảng Android và iOS và đã được đăng tải lên các cửa hàng trên iOS (App Store) (Xem hình \ref{fig:app-store}) và Android (Google Play) (Xem hình \ref{fig:google-play}). Xem và tải ứng dụng trên Android tại đây \footnote{Xem thêm về ứng dụng trên Andoird tại đây:\url{https://bitly.com.vn/5a5xax}} và trên iOS tại đây \footnote{Xem thêm về ứng dụng trên iOS tại đây:\url{https://bitly.com.vn/7gcxli}}

\begin{figure}[H]
    \centering
    \includegraphics[width=5cm]{images/GooglePlay.jpg}
    \caption{Ứng dụng trên cửa hàng Google Play của Android}
    \label{fig:google-play}
\end{figure}

\begin{figure}[H]
    \centering
    \includegraphics[width=5cm]{images/AppStore.jpg}
    \caption{Ứng dụng trên cửa hàng App Store của iOS}
    \label{fig:app-store}
\end{figure}

\subsection{Các chức năng chính của ứng dụng}

Ứng dụng chatbot được sử dụng với ngôn ngữ chính là tiếng Viêt, dữ liệu đầu vào và đầu ra bao gồm giọng nói và văn bản được xây dựng trong lĩnh vực hỏi đường đi trong phạm vi Thành phố Thủ Đức.
\begin{itemize}
    \item[--] Hỏi đường bằng văn bản: Người dùng nhập 1 câu hỏi bằng văn bản vào ứng dụng. VD: “Đường đi từ UBND thành phố Thủ Đức đến Công An thành phố Thủ Đức”. Bấm nút gửi
    \item[--] Hỏi đường bằng giọng nói: Người dùng nhấn icon ghi âm và nói một câu vào ứng dụng. VD: “Đường đi từ UBND thành phố Thủ Đức đến Công An thành phố Thủ Đức”
    \item[--] Ứng dụng hiển thị câu trả lời bằng văn bản lên màn hình và chuyển văn bản thành âm thanh phát lên.
    \item[--] Ở màn hình bản đồ, bạn có thể xem được đường đi cụ thể mà bạn vừa yêu cầu và khoảng cách sẽ được hiển thị lên bản đồ. Bạn có thể phóng to, thu nhỏ bản đồ và có thể xem vị trí hiện tại của mình.
\end{itemize}
\textbf{Ứng dụng được xây dựng dựa trên các ý định:}
\begin{itemize}
    \item[--] Hỏi đường đi từ một địa điểm đến một địa điểm
    \item[--] Hỏi khoảng cách từ một địa điểm đến một địa điểm
    \item[--] Hỏi vị trí hiện tại của người dùng
    \item[--] Hỏi địa chỉ của một địa điểm  
\end{itemize}
\textbf{Giao diện chức năng hội thoại:}

Giao diện ứng dụng khi mở lên sẽ hiển thị như sau (hình \ref{fig:home}):
\begin{figure}[H]
    \centering
    \includegraphics[width=5cm]{images/home.jpg}
    \caption{Màn hình trò chuyện}
    \label{fig:home}
\end{figure}
Khi nhấn vào nút "Nói chuyện", màn hình sẽ chuyển sang màn hình cuộc hội thoại.
Giao diện ứng dụng khi thực hiện các đoạn hội thoại giữa người và máy sẽ hiển thị như sau (hình \ref{fig:screen-chat}) :
\begin{figure}[H]
    \centering
    \includegraphics[width=5cm]{images/Screen-chat.jpg}
    \caption{Màn hình trò chuyện}
    \label{fig:screen-chat}
\end{figure}

Giao diện gồm các phần sau đây:
\begin{itemize}
    \item[--] Thanh nhập dữ liệu để người dùng nhập văn bản, một nút ghi âm dùng để ghi âm giọng nói, một nút bấm để thực hiện gửi yêu cầu từ người dùng đến hệ thống.
    \item[--] Bên phải khung hình hiển thị dữ liệu người dùng nhập vào.
    \item[--] Bên trái khung hình hiển thị trả lời từ ứng dụng.
    \item[--] Thời gian tin nhắn được gửi đi và thời gian nhận được phản hồi.
    \item[--] Khi tìm kiếm kết quả thành công, kết quả hướng dẫn đường đi được trả về, nút "Xem đường đi" dùng để chuyển sang màn hình bản đồ chỉ đường.
\end{itemize}

Khi người dùng bấm vào icon "micro" trên màn hình cuộc hội thoại, nếu đây là lần đầu sử dụng ứng dụng với chức năng này, ứng dụng sẽ xin người dùng cấp quyền cho phép ghi âm từ thiết bị (xem hình \ref{fig: accesss-mic}), khi chọn cho phép, người dùng có thể sử dụng chức năng này để ghi âm giọng nói.

\begin{figure}[H]
    \centering
    \includegraphics[width=5cm]{images/access_mic.jpg}
    \caption{Màn hình xin quyền cấp phép ghi âm của thiết bị}
    \label{fig: accesss-mic}
\end{figure}

Giao diện ứng dụng khi đang thực hiện ghi âm giọng nói sẽ hiển thị như sau (hình \ref{fig:screen-record}) :

\begin{figure}[H]
    \centering
    \includegraphics[width=5cm]{images/Screen-record.png}
    \caption{Màn hình thể hiện đang ghi âm}
    \label{fig:screen-record}
\end{figure}

Bán kính của vòng tròn hiển thị trong khi thực hiện quá trình ghi âm được hiển thị tăng giảm theo cường độ âm thanh thu được.

Sau khi kết thúc yêu cầu bằng giọng nói, ứng dụng tự động chuyển đổi giọng nói thành văn bản gửi lên hệ thống và hiển thị văn bản lên màn hình cuộc hội thoại.


\textbf{Giao diện chức năng hiển thị bản đồ:}

Nếu đây là lần đầu sử dụng ứng dụng với chức năng này, ứng dụng sẽ xin người dùng cấp quyền cho phép truy cập vào vị trí của thiết bị (xem hình \ref{fig: accesss-location}). Người dùng cần phải cho phép để ứng dụng có thể biết được vị trí để thuận tiện hơn cho quá trình sử dụng chức năng này của ứng dụng.

\begin{figure}[H]
    \centering
    \includegraphics[width=5cm]{images/access_location.jpg}
    \caption{Màn hình xin quyền cấp phép truy cập vị trí hiện tại của thiết bị}
    \label{fig: accesss-location}
\end{figure}

Giao diện ứng dụng khi người dùng xem bản đồ đường đi trên ứng dụng (hình \ref{fig:screen-map}) :
\begin{figure}[H]
    \centering
    \includegraphics[width=5cm]{images/Screen-Map.png}
    \caption{Màn hình chỉ đường}
    \label{fig:screen-map}
\end{figure}

Giao diện gồm các phần sau đây:
\begin{itemize}
    \item[--] "Bắt đầu": Hiển thị thông tin địa điểm bắt đầu.
    \item[--] "Kết thúc": Hiển thị thông tin địa điểm kết thúc.
    \item[--] Hiển thị khoảng cách của hai địa điểm.
    \item[--] Chính giữa màn hình hiển thị bản đồ, điểm đầu và điểm cuối đường đi và đường đi giữa hai điểm được thể hiện bằng dòng kẻ màu đỏ.
    \item[--] Bên trái màn hình có nút phóng to và thu nhỏ bản đồ.
    \item[--] Bên góc bên phải phía dưới màn hình có nút dùng để xác định vị trí hiện tại của người dùng.
\end{itemize}