\chapter{Kết luận và Hướng phát triển}
\label{Chapter7}

\emph{Chương này trình bày kết luận sau quá trình thực hiện đề tài, bao gồm kỹ năng nghiên cứu, kỹ năng xây dựng một ứng dụng hoàn chỉnh, kết quả tóm tắt, ý nghĩa và khuyết điểm của đề tài. Cuối cùng là hướng phát triển và những định hướng trong tương lai.}

\section{Kết luận}
\label{sec:ket-luan}

\subsection{Tóm tắt kết quả đạt được}

Trong quá trình thực hiện khóa luận, nhóm em đã học hỏi và đạt được một số kết quả. Cụ thể:

\begin{itemize}
    \item[--] Về mặt nghiên cứu và ứng dụng:
        \begin{itemize}
            \item[\textbullet] Nghiên cứu phương pháp xác định ý định và trích xuất thực thể (Detect Intent \& Extract Entities) từ văn bản nhận được bằng cách ứng dụng thư viện mã nguồn mở Snips NLU. 
            \item[\textbullet] Nghiên cứu ứng dụng phương pháp chuyển đổi giọng nói thành văn bản và chuyển đổi văn bản thành giọng nói với bộ thư viện speech\_to\_text và flutter\_tts.  
            \item[\textbullet] Tìm kiếm đưa ra câu trả lời phù hợp (Search compatible response) bằng cách sử dụng Google Map API để tìm kiếm kết quả trả về phù hợp. 
            \item[\textbullet] Tạo bộ dữ liệu huấn luyện bao gồm 80 câu huấn luyện và 40 câu kiểm thử cho 4 loại ý định trong lĩnh vực tìm đường đi. 
            \item[\textbullet] Xây dựng thành công ứng dụng chatbot chỉ đường trong phạm vi Thành phố Thủ Đức, với giao diện đơn giản, dễ dàng sử dụng và ngôn ngữ sử dụng chính là tiếng Việt. 
            \item[\textbullet] Ứng dụng được xây dựng và phát triển trên cả hai nền tảng là Android và iOS.
        \end{itemize}
    \item[--] Về mặt kĩ năng mềm
        \begin{itemize}
            \item[\textbullet] Kỹ năng đọc tài liệu và phân tích vấn đề.
            \item[\textbullet] Kỹ năng nghiên cứu dự án.
            \item[\textbullet] Kĩ năng làm việc nhóm và phân chia công việc.
            \item[\textbullet] Kĩ năng trình bày và thuyết trình, viết báo cáo.
        \end{itemize}
\end{itemize}

\subsection{Ý nghĩa}
Bên cạnh những kết quả, việc hoàn thành đề tài Xây dựng giải pháp trả lời tự động (chatbot) bằng tiếng Việt cũng mang lại những ý nghĩa đáng kể:

\begin{itemize}
    \item[--] Kết quả của đề tài là minh chứng cho tính khả thi của đề tài.
    \item[--] Đề tài cũng chứng minh ý nghĩa của một chatbot trong đời sống. Với việc tồn tại một chatbot có khả năng nghe hiểu và phản hồi bằng giọng nói và văn bản bằng tiếng Việt, rất nhiều vấn đề sẽ trở nên đơn giản, dễ dàng hơn và giúp tiết kiệm thời gian cho người dùng.
\end{itemize}

\subsection{Những hạn chế, giới hạn}
Ứng dụng chatbot giọng nói bằng tiếng Việt hoạt động tương đối tốt. Tuy nhiên, vẫn còn nhiều mặt hạn chế nhất định như sau:
\begin{itemize}
    \item[--] Đối với vấn đền phân tích intent và trích xuất slot đôi khi vẫn còn nhầm lẫn và chưa được chính xác.
    \item[--] Đối với từ điển dịch tự tạo còn ít từ và nhiều hạn chế khi dịch.
    \item[--] Đối với vấn đề phân tích văn bản thành giọng nói còn hạn chế ngôn ngữ tiếng Việt trên iOS.
\end{itemize}

\section{Hướng phát triển}


Từ những hạn chế đã được nêu phần Kết luận \ref{sec:ket-luan}, trong tương lai, nhóm có những dự định bao gồm cải tiến thuật toán phân tích intent, trích xuất slot, dịch thuật và bổ sung thêm những chức năng, câu hỏi đa dạng khác ngoài những câu hỏi đơn giản như hiện tại. Cụ thể như:

Đối với vấn đề phân tích intent và trích xuất slot, nhóm sẽ tiếp tục tìm các giải pháp khác nhau để cải thiện độ chính xác khi phân tích dữ liệu. Bằng cách thực hiện huấn luyện mô hình nhiều hơn và đào sâu vào thuật toán để xem xét nguyên nhân cụ thể và đưa ra hướng giải quyết.

Đối với chức năng các câu hỏi chỉ dẫn, bổ sung thêm một số câu hỏi hữu ích khác như tra cứu các địa điểm quán ăn, cây xăng xung quanh vị trí hiện tại hay ở một khu vực nhất định, tìm nhiều đường đi khác nhau, hỏi và phân tích được vấn đề giao thông hiện tại như kẹt xe,... Và nâng cao hơn là áp dụng các mô hình xử lý ngôn ngữ tự nhiên để hệ thống có thể hiểu câu lệnh tốt hơn.