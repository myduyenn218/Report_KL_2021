\chapter{Khảo sát}
\label{Chapter2}

\emph{Chương này giới thiệu về một số hệ thống tương tự với hệ thống chatbot cũng như một số thuật toán xử lí hiểu ngôn ngữ tự nhiên.}

\section{Chatbot}
\subsection{Chatbot là gì?}

Chatbot - hay còn được gọi là chatterbot - là một ứng dụng phần mềm được sử dụng để thực hiện một cuộc trò chuyện trực tuyến thông qua văn bản hoặc chuyển văn bản thành giọng nói, thay cho việc cung cấp liên hệ trực tiếp với một nhân viên trực tiếp. Trợ lý ảo chatbot đang ngày càng được sử dụng để xử lý các tác vụ đơn giản, tra cứu trong cả môi trường doanh nghiệp với người tiêu dùng và doanh nghiệp với doanh nghiệp.

Chatbot có thể có nhiều mức độ phức tạp khác nhau, không trạng thái hoặc trạng thái. Một chatbot không trạng thái tiếp cận mỗi cuộc trò chuyện như thể nó đang tương tác với một người dùng mới. Ngược lại, một chatbot trạng thái có thể xem xét các tương tác trong quá khứ và định khung các phản hồi mới theo ngữ cảnh\cite{chat-bot}.

Chatbot có thể được phân chia thành các loại sau đây:
\begin{itemize}
    \item Chatbot có kịch bản hoặc trả lời nhanh - Đây là những chatbot cơ bản nhất; chúng hoạt động như một cây quyết định phân cấp. Các bot này tương tác với người dùng thông qua một tập hợp các câu hỏi được xác định trước sẽ tiến triển cho đến khi chatbot trả lời câu hỏi của người dùng. Tương tự như chatbot này là chatbot dựa trên menu yêu cầu người dùng thực hiện các lựa chọn từ danh sách hoặc menu được xác định trước để cung cấp cho bot hiểu sâu hơn về những gì khách hàng đang tìm kiếm.

    \item Chatbot dựa trên nhận dạng từ khóa - Những chatbot này phức tạp hơn một chút; chúng cố gắng lắng nghe những gì người dùng nhập và phản hồi tương ứng bằng cách sử dụng các từ khóa chọn được từ phản hồi của khách hàng. Các từ khóa có thể tùy chỉnh và \ac{ai} được kết hợp trong bot này để cung cấp phản hồi thích hợp cho người dùng. Những chatbot này thường gặp khó khăn khi phải đối mặt với việc sử dụng từ khóa lặp đi lặp lại hoặc các câu hỏi thừa.

    \item Chatbot kết hợp - Những chatbot này kết hợp các yếu tố của bot dựa trên menu và nhận dạng từ khóa. Người dùng có thể chọn để câu hỏi của họ được trả lời trực tiếp, nhưng cũng có thể truy cập menu của chatbot để thực hiện lựa chọn nếu quá trình nhận dạng từ khóa tạo ra kết quả không hiệu quả.

    \item Chatbot theo ngữ cảnh - Những chatbot này phức tạp hơn những chatbot được liệt kê ở trên và yêu cầu tập trung vào dữ liệu. Họ sử dụng Machine Learning và \ac{ai} để ghi nhớ các cuộc trò chuyện và tương tác với người dùng, sau đó sử dụng những ký ức này để phát triển và cải thiện theo thời gian. Thay vì dựa vào từ khóa, những bot này sử dụng những gì khách hàng yêu cầu và cách chúng yêu cầu để đưa ra câu trả lời và tự cải thiện.

    \item Chatbot hỗ trợ giọng nói - Loại chatbot này là tương lai của công nghệ chatbot. Các chatbot hỗ trợ giọng nói sử dụng đối thoại bằng giọng nói từ người dùng làm dữ liệu đầu vào. Chúng có thể được tạo ra bằng cách sử dụng bộ công cụ chuyển đổi văn bản thành giọng nói (\ac{tts}) và giao diện ứng dụng nhận dạng giọng nói (\ac{api}) \cite{chat-bot}.

\end{itemize}
Ưu điểm của chatbot:
\begin{itemize}
    \item Cung cấp dịch vụ khách hàng nhanh chóng hơn:
          \begin{itemize}
              \item[--] Phần mềm này hỗ trợ doanh nghiệp cung cấp dịch vụ khách hàng 24 giờ/ngày, bất kể cuối tuần hay nghỉ lễ.
              \item[--] Khi khách hàng trực tuyến có thắc mắc, họ chỉ cần hỏi trong chatbot trên trang web của chúng ta mà không cần phải chờ đợi lâu để có câu trả lời. Bởi câu trả lời chỉ là một vài tổ hợp được lập trình sẵn.
          \end{itemize}
    \item Làm tăng sự hài lòng của khách hàng:
          \begin{itemize}
              \item[--] Khi khách hàng nhận được câu trả lời thỏa đáng với dịch vụ nhanh chóng nhờ chatbot, họ sẽ cảm thấy hài lòng hơn và tiếp tục mua sản phẩm của chúng ta.
          \end{itemize}
    \item Giảm chi phí lao động:
          \begin{itemize}
              \item[--] Chatbot giúp bạn giữ chi phí kinh doanh thấp bởi số tiền bạn đầu tư vào chatbot ít hơn số tiền bạn phải trả cho nhân viên.
              \item[--] Bằng cách này, bạn thực sự tiết kiệm được rất nhiều tiền thay vì việc duy trì một trung tâm hỗ trợ khách hàng. Tính năng này sẽ giúp bạn tiết kiệm tài chính, tránh những rắc rối trong quản lý nhân sự, và tiết kiệm thời gian để làm những việc cần thiết khác.
              \item[--] Làm việc 24/7. Đáp ứng cho giải pháp làm việc toàn cầu (golbal), khác múi giờ. 
          \end{itemize}
    \item Nhiều mục đích sử dụng:
          \begin{itemize}
              \item[--] Bạn có thể sử dụng chatbot trong nhiều mảng, ví dụ như nhận đơn đặt hàng của khách, dịch vụ khách hàng và quảng cáo sản phẩm.
          \end{itemize}
\end{itemize}
\subsection{Các thành phần của chatbot}
\begin{figure}[htp]
    \centering
    \includegraphics[width=15cm]{images/Kiến trúc cơ bản của hệ thống giao tiếp tự động.png}
    \caption{Các thành phần cơ bản của chatbot}
    \label{fig:Kiến trúc cơ bản của hệ thống giao tiếp tự động}
\end{figure}
Một hệ thống chatbot bao gồm các thành phần sau: Bộ giải mã đầu vào (Input Decoder), hiểu ngôn ngữ tự nhiên (Natural Language Understanding \ac{nlu}), trình quản lý hội thoại (Dialogue Manager), sinh ngôn ngữ tự nhiên (Natural Language Generation \ac{nlg}) và trình kết xuất đầu ra (Output Renderer). Xem hình \ref{fig:Kiến trúc cơ bản của hệ thống giao tiếp tự động}.

\begin{itemize}
  \item[--] Bộ giải mã đầu vào (Input Decoder): thành phần này dùng để chuyển đổi input thành văn bản. Input ở đây có thể là giọng nói, cử chỉ hoặc chữ viết tay. 
    \item[--] Hiểu ngôn ngữ tự nhiên (Natural Language Understanding \ac{nlu}): Thành phần này đóng vai trò rất quan trọng trong hệ thống chatbot. Nó trích ra các thông tin cần thiết từ input để thành phần Dialogue Manager sử dụng. Hai thành phần không thể thiếu của \ac{nlu} là bộ phân loại ý định và \ac{ner}. 
        \item[--] Trình quản lý hội thoại (Dialogue Manager): thành phần này giúp quản lý luồng hội thoại. Dựa vào ngữ cảnh và dữ liệu từ \ac{nlu}, thành phần Dialogue Manager sẽ thực hiện truy vấn xuống cơ sở dữ liệu, hiển thị thông báo lỗi nếu có lỗi, hoặc trong trường hợp chưa hiểu ý rõ ý định của người dùng nó có thể tạo câu hỏi để biết thêm thông tin.
        \item[--] Sinh ngôn ngữ tự nhiên (Natural Language Generation \ac{nlg}): thành phần này giúp tạo câu trả lời để trả về cho người dùng. Đối với hệ thống đơn giản chúng ta có thể định nghĩa trước những câu trả lời thay vì sử dụng \ac{nlg}.
         \item[--] Trình kết xuất đầu ra (Output Renderer): thành phần này giúp biểu diễn câu trả lời cho người dùng. Câu trả lời có thể hiển thị dưới nhiều dạng như văn bản, âm thanh,...

\end{itemize}

Bài luận này sẽ nghiên cứu về thành phần hiểu ngôn ngữ tự nhiên (\ac{nlu}) cho việc phát hiện ý định (intent) và trích xuất thực thể (entity).

\section{Các hệ thống tương tự}

Hiện nay, việc xây dựng chatbot trong nhiều lĩnh vực như kinh doanh, y tế, giải trí,... khá là phát triển và đem lại được nhiều lợi ích lớn đối với người dùng. Trong lĩnh vực chỉ đường, chúng em nhận thấy rằng hiện cũng có một số chatbot được xây dựng như Google Assistant, Seri và cũng có không ít các ứng dụng dùng để chỉ đường như Google Maps\cite{ggmaps},...

Google Maps là một dịch vụ bản đồ web được phát triển bởi Google. Nó cung cấp hình ảnh vệ tinh, ảnh chụp trên không, bản đồ đường phố, chế độ xem toàn cảnh tương tác 360° của đường phố, điều kiện giao thông thời gian thực và lập kế hoạch tuyến đường để đi bộ, xe hơi, xe đạp và máy bay hoặc giao thông công cộng. Năm 2020, Google Maps được sử dụng bởi hơn 1 tỷ người mỗi tháng trên thế giới\cite{wiki-ggmaps}.

Google Maps khởi đầu được thiết kế bởi hai anh em người Đan Mạch, tại công ty Where 2 Technologies có trụ sở tại Sydney. Lần đầu tiên nó được thiết kế để người dùng tải xuống riêng biệt, nhưng sau đó công ty đã trình bày ý tưởng về một sản phẩm hoàn toàn dựa trên Web cho ban quản lý của Google, thay đổi phương thức phân phối. Vào tháng 10 năm 2004, công ty được mua lại bởi Google - nơi nó chuyển thành ứng dụng web Google Maps. Sản phẩm hiện đã xuất hiện ở cả hai phiên bản web và mobile.

Google Map hiện giờ rất phổ biến đối với người dùng và nó có rất nhiều tính năng về vị trí, trong đó có thể nói tính năng chỉ dẫn đường đi là vô cùng phổ biến với người dùng.

Tính năng chỉ đường của Google Maps cung cấp công cụ lập kế hoạch tuyến đường, cho phép người dùng tìm chỉ đường khả dụng thông qua lái xe, giao thông công cộng, đi bộ hoặc đi xe đạp.

Ứng dụng Google Maps trên điện thoại cũng tích hợp sử dụng thu âm, chuyển giọng nói thành văn bản khi nhập điểm bắt đầu và điểm kết thúc để người dùng có thể dễ nhập điểm đầu và cuối.

\begin{figure}[H]
    \centering
    \includegraphics[width=10cm]{images/HomePage-GoogleMaps.png}
    \caption{Trang chủ Google Maps}
    \label{fig:homepage-ggmaps}
\end{figure}

Android Auto là một ứng dụng di động được phát triển bởi Google nhằm đưa các tính năng từ một thiết bị Android (ví dụ như điện thoại thông minh) lên hệ thống bảng thông báo và giải trí tương thích trên xe hơi.

\begin{figure}[H]
    \centering
    \includegraphics[width=10cm]{images/Android-Auto.png}
    \caption{Trang chủ Android Auto}
    \label{fig:homepage-android-auto}
\end{figure}

Kết nối điện thoại với màn hình ô tô — ứng dụng Android sẽ hiển thị trên màn hình hiển thị của xe. Nhấn để nhận chỉ đường lái xe hoặc nói chuyện để gửi tin nhắn; thậm chí bạn có thể dùng nó để gọi điện thoại khi rảnh tay. Android Auto được tạo ra để giúp bạn tập trung hơn trong quá trình lái xe.

\begin{figure}[H]
    \centering
    \includegraphics[width=10cm]{images/widescreen.png}
    \caption{Xem mọi thứ trên màn hình rộng.}
    \label{fig:homepage-widescreen-Android}
\end{figure}

Với mong muốn xây dựng một ứng dụng chatbot sử dụng trong lĩnh vực chỉ đường, chúng em xây dựng một ứng dụng tiếng Việt, trong đó có sử dụng giọng nói để tương tác với ứng dụng, hỏi đường chỉ bằng một câu nói dễ dàng, sử dụng được mọi lúc, một cách đơn giản mà không làm mất đi sự tập trung của người dùng khi lái xe.

\section{Các nền tảng hiểu ngôn ngữ tự nhiên}

Hiểu ngôn ngữ tự nhiên (\ac{nlu}) là thành phần rất quan trọng và không thể thiếu trong chatbot cũng như trợ lý ảo. Nó giúp chúng ta nhận diện ý định (intent) và các thực thể (entity) từ câu nói của người dùng.

Hiện nay có rất nhiều nền tảng cung cấp thành phần hiểu ngôn ngữ tự nhiên (\ac{nlu}), trong số đó những dịch vụ hiểu ngôn ngữ tự nhiên (\ac{nlu}) phổ biến nhất là: LUIS\footnote{\url{https://www.luis.ai/}}, Watson\footnote{\url{https://www.ibm.com/watson}}, DialogFlow\footnote{\url{https://cloud.google.com/dialogflow/docs/}}, Rasa\footnote{\url{https://rasa.com/}} và Snips\footnote{\url{https://snips.ai/}}.


\subsection{Yêu cầu lựa chọn}
Do sự tồn tại của nhiều dịch vụ hiểu ngôn ngữ tự nhiên (\ac{nlu}), việc lựa chọn ra một nền tảng để sử dụng cũng rất khó khăn. Nhóm em lựa chọn nền tảng hiểu ngôn ngữ tự nhiên (\ac{nlu}) dựa trên các tiêu chí sau đây:

\begin{itemize}
  \item[--] Độ chính xác: nền tảng có khả năng trích xuất ý định và các thực thể với độ chính xác lớn hơn 80\%.
  \item[--] Thời gian: thời gian dự đoán nhanh chóng.
    \item[--] Chi phí:  tốn ít hoặc không tốn chi phí để sử dụng.
    \item[--] Bảo mật: Những dữ liệu cá nhân bảo mật tốt, không được truy cập dữ liệu một cách trái phép.
    \item[--] Độ tin cậy: Phần mềm có độ tin cậy cao, ít xảy ra lỗi.
    \item[--] Tính linh hoạt: dễ dàng chỉnh sửa để phù hợp với nhu cầu sử dụng. 
\end{itemize}

\subsection{Tóm tắt các bài so sánh được sử dụng}

Do thời gian thực hiện nghiên cứu có giới hạn, nhóm em đã sử dụng kết quả của các bài báo so sánh hiệu năng của các nền tảng hiểu ngôn ngữ tự nhiên (\ac{nlu}) bao gồm:

\begin{enumerate}
    \item Đánh giá các dịch vụ hiểu ngôn ngữ tự nhiên cho các hệ thống trả lời câu hỏi hội thoại (Evaluating Natural Language Understanding Services for Conversational Question Answering Systems). \cite{EvaluatingNLU}.
    \item Đo điểm chuẩn các dịch vụ hiểu ngôn ngữ tự nhiên cho việc xây dựng hệ thống hội thoại (Benchmarking Natural Language Understanding Services for building Conversational Agents). \cite{BenchmarkingNLU}.
    \item Nền tảng giọng nói Snips: một hệ thống hiểu ngôn ngữ tự nhiên nhúng cho các giao diện giọng nói riêng tư theo thiết kế (Snips Voice Platform: an embedded Spoken Language Understanding system for private-by-design voice interfaces).\cite{snips-nlu}.
\end{enumerate}

\textbf{I. Đánh giá các dịch vụ hiểu ngôn ngữ tự nhiên cho các hệ thống trả lời câu hỏi hội thoại.}

Bài báo so sánh hiệu năng giữa các dịch vụ hiểu ngôn ngữ tự nhiên (\ac{nlu}) khác nhau bao gồm: LUIS, Watson, API.ai và RASA.

\textbf{Dữ liệu:}

Dựa trên 2 kho dữ liệu khác nhau:
\begin{itemize}
    \item[--] Kho ngữ liệu chatbot (The Chatbot Corpus): Bộ dữ liệu các câu hỏi về kết nối giao thông công cộng.
    \item[--] Kho ngữ liệu StackExchange (The StackExchange Corpus): dựa trên dữ liệu từ hai nền tảng StackExchange\footnote{\url{https://stackexchange.com/}}: hỏi ubuntu (ask ubuntu\footnote{\url{https://askubuntu.com/}}) và ứng dụng web (Web Applications\footnote{\url{https://webapps.stackexchange.com/}}). Cả hai kho tài liệu đều có sẵn trên GitHub\footnote{\url{https://github.com/sebischair/NLU-Evaluation-Corpora}}.
\end{itemize}

Kho ngữ liệu chatbot (Chatbot Corpus):

\begin{itemize}
    \item[--] Bao gồm 206 câu hỏi được gán nhãn thủ công. Bao gồm 2 ý định (intent) "thời gian rời ga" (Departure Time), "tìm tuyến đi" (Find Connection) và 5 loại thực thể (entity) khác nhau "trạm bắt đầu" (StationStart), "trạm đích"  (StationDest), "tiêu chuẩn" (Criterion), "phương tiện" (Vehicle), "tuyến" (Line). Ngôn ngữ chủ yếu là tiếng Anh, một số tên trạm và tên đường bằng tiếng Đức. Dữ liệu được chia thành làm 2 phần, dữ liệu huấn luyện (training) 100 câu và dữ liệu kiểm thử (test) 106 câu. Chi tiết xem bảng \ref{fig:comparisonimg-1}.
\end{itemize}


\begin{table}[]
\begin{center}
\begin{tabular}{|l|l|l|l|}
\hline
\textbf{\begin{tabular}[c]{@{}l@{}}Loại thực thể\\ (Entity type)\end{tabular}} & \textbf{\begin{tabular}[c]{@{}l@{}}Huấn luyện\\ (Training)\end{tabular}} & \textbf{\begin{tabular}[c]{@{}l@{}}Kiểm thử\\ (Test)\end{tabular}} & $\Sigma$ \\ \hline
\begin{tabular}[c]{@{}l@{}}Trạm bắt đầu\\ (StationStart)\end{tabular}          & 91                                                                       & 102                                                                & 193                   \\ \hline
\begin{tabular}[c]{@{}l@{}}Trạm đích\\ (StationDest)\end{tabular}              & 57                                                                       & 71                                                                 & 128                   \\ \hline
\begin{tabular}[c]{@{}l@{}}Tiêu chuẩn\\ (Criterion)\end{tabular}               & 48                                                                       & 34                                                                 & 82                    \\ \hline
\begin{tabular}[c]{@{}l@{}}Phương tiện\\ (Vehicle)\end{tabular}                & 50                                                                       & 35                                                                 & 85                    \\ \hline
\begin{tabular}[c]{@{}l@{}}Tuyến\\ (Line)\end{tabular}                         & 4                                                                        & 2                                                                  & 6                     \\ \hline
$\Sigma$                                                          & 250                                                                      & 244                                                                & 494                   \\ \hline
\end{tabular}
\caption{Các loại thực thể (entity) trong kho ngữ liệu chatbot}
\label{fig:comparisonimg-1}
\end{center}
\end{table}

Kho ngữ liệu StackExchange (StackExchange Corpus):

\begin{itemize}
    \item[--] Đối với bộ dữ liệu StackExchange, tác giả chọn ra những câu hỏi có đánh giá cao nhất và nhiều lượt xem nhất từ 2 nền tảng hỏi ubuntu (ask ubuntu) và ứng dụng web (web applications).
        \item[--]Bộ dữ liệu thu được tổng cộng là 290 câu hỏi, 100 câu từ ứng dung web (web applications) và 190 câu từ hỏi ubuntu (ask ubuntu).
        \item[--]Ý định (intent) và thực thể (entity) cũng được gán nhãn thủ công:
        \begin{itemize}
            \item[+]Đối với ask ubuntu, bao gồm các ý định   "cập nhật" (Make Update), "cài đặt máy in" (Setup Printer), "tắt máy tính" (Shutdown Computer), và  "Khuyến nghị phần mềm" (Software Recommendation). Những thực thể có thể có là : "Tên phần mềm" (SoftwareName), "máy in" (Printer), và "phiên bản Ubuntu" (UbuntuVersion).
            \item[+] Đối với ứng dụng web (web applications), bao gồm các ý định "đổi mật khẩu" (Change Password), "xóa tài khoản" (Delete Account), "tải video" (Download Video), "xuất dữ liệu" (Export Data), "lọc spam"  (Filter Spam), "Tìm thay thế" (Find Alternative), và "đồng bộ tài khoản" (Sync Accounts). Những thực thể có thể có là "dịch vụ web" (WebService), "Hệ điều hành" (OperatingSystem) và "trình duyệt" (Browser).
        \end{itemize}
    \item[--] Sau khi gán nhãn và chọn lọc, bộ dữ liệu cuối cùng bao gồm 251 câu, trong đó 162 câu thuộc hỏi ubuntu (ask ubuntu) và 89 câu thuộc ứng dụng web (web applications).
    \item[--] Dữ liệu ứng dụng web (web applications) được chia thành dữ liệu huấn luyện (training) và dữ liệu kiểm thử (test) như bảng  \ref{fig:comparisonimg-webappdatasets}.
     \item[--] Dữ liệu hỏi ubuntu (ask ubuntu) được chia thành dữ liệu huấn luyện (training) và dữ liệu kiểm thử (test) như bảng \ref{fig:comparisonimg-askubuntudatasets}.
      \item[--] Bảng \ref{fig:comparisonimg-entityTypesStackExchange} mô tả các thực thể (entity) có trong bộ ngữ liệu StackExchange.
\end{itemize}


\begin{table}[]
 \begin{center}
\begin{tabular}{|l|l|l|l|}
\hline
\textbf{\begin{tabular}[c]{@{}l@{}}Ý định\\ (Intent)\end{tabular}}         & \textbf{\begin{tabular}[c]{@{}l@{}}Huấn luyện\\ (training)\end{tabular}} & \textbf{\begin{tabular}[c]{@{}l@{}}Kiểm thử\\ (test)\end{tabular}} & $\Sigma$ \\ \hline
\begin{tabular}[c]{@{}l@{}}Đổi mật khẩu\\ (ChangePassword)\end{tabular}    & 2                                                                        & 6                                                                  & 8        \\ \hline
\begin{tabular}[c]{@{}l@{}}Xóa tài khoản\\ (DeleteAccount)\end{tabular}    & 7                                                                        & 10                                                                 & 17       \\ \hline
\begin{tabular}[c]{@{}l@{}}Tải video\\ (DownloadVideo)\end{tabular}        & 1                                                                        & 0                                                                  & 1        \\ \hline
\begin{tabular}[c]{@{}l@{}}Xuất dữ liệu\\ (ExportData)\end{tabular}        & 2                                                                        & 3                                                                  & 5        \\ \hline
\begin{tabular}[c]{@{}l@{}}Lọc spam\\ (FilterSpam)\end{tabular}            & 6                                                                        & 14                                                                 & 20       \\ \hline
\begin{tabular}[c]{@{}l@{}}Tìm thay thế\\ (FindAlternative)\end{tabular}   & 7                                                                        & 16                                                                 & 23       \\ \hline
\begin{tabular}[c]{@{}l@{}}Đồng bộ tài khoản\\ (SyncAccounts)\end{tabular} & 3                                                                        & 6                                                                  & 9        \\ \hline
None                                                                       & 2                                                                        & 4                                                                  & 6        \\ \hline
$\Sigma$                                                                   & 30                                                                       & 54                                                                 & 84       \\ \hline
\end{tabular}
\caption{Bộ dữ liệu web applications}
    \label{fig:comparisonimg-webappdatasets}
    \end{center}
\end{table}


\begin{table}[]
    \begin{center}
\begin{tabular}{|l|l|l|l|}
\hline
\textbf{\begin{tabular}[c]{@{}l@{}}Ý định\\ Intent\end{tabular}}         & \textbf{\begin{tabular}[c]{@{}l@{}}Huấn luyện\\ (training)\end{tabular}} & \textbf{\begin{tabular}[c]{@{}l@{}}Kiểm thử\\ (test)\end{tabular}} & $\Sigma$ \\ \hline
\begin{tabular}[c]{@{}l@{}}Cập nhật\\ (MakeUpdate)\end{tabular}          & 10                                                                       & 37                                                                 & 47       \\ \hline
\begin{tabular}[c]{@{}l@{}}Cài đặt máy in\\ (SetupPrinter)\end{tabular}  & 10                                                                       & 13                                                                 & 23       \\ \hline
\begin{tabular}[c]{@{}l@{}}Tắt máy tính\\ (ShutdownCompute)\end{tabular} & 13                                                                       & 14                                                                 & 27       \\ \hline
\begin{tabular}[c]{@{}l@{}}Khuyến nghị\\ (Recommendation)\end{tabular}   & 17                                                                       & 40                                                                 & 57       \\ \hline
None                                                                     & 3                                                                        & 5                                                                  & 8        \\ \hline
$\Sigma$                                                                 & 53                                                                       & 109                                                                & 162      \\ \hline
\end{tabular}
 \caption{Bộ dữ liệu ask ubuntu}
    \label{fig:comparisonimg-askubuntudatasets}
    \end{center}
\end{table}



\begin{table}[]
    \begin{center}
\begin{tabular}{|l|l|l|l|l|}
\hline
\textbf{\begin{tabular}[c]{@{}l@{}}Bộ dữ liệu\\ (dataset)\end{tabular}}            & \textbf{\begin{tabular}[c]{@{}l@{}}Loại thực thể\\ (Entity Type)\end{tabular}} & \textbf{\begin{tabular}[c]{@{}l@{}}Huấn luyện\\ (training)\end{tabular}} & \textbf{\begin{tabular}[c]{@{}l@{}}Kiểm thử\\ (test)\end{tabular}} & $\Sigma$ \\ \hline
\multirow{4}{*}{\begin{tabular}[c]{@{}l@{}}Ứng dụng web\\ (Web apps)\end{tabular}} & \begin{tabular}[c]{@{}l@{}}Dịch vụ web\\ (WebService)\end{tabular}             & 33                                                                       & 64                                                                 & 97       \\ \cline{2-5} 
                                                                                   & \begin{tabular}[c]{@{}l@{}}Hệ điều hành\\ (OS)\end{tabular}                    & 1                                                                        & 0                                                                  & 1        \\ \cline{2-5} 
                                                                                   & \begin{tabular}[c]{@{}l@{}}Trình duyệt\\ (Browser)\end{tabular}                & 1                                                                        & 0                                                                  & 1        \\ \cline{2-5} 
                                                                                   & $\Sigma$                                                                       & 35                                                                       & 64                                                                 & 99       \\ \hline
\multirow{4}{*}{\begin{tabular}[c]{@{}l@{}}Hỏi Ubuntu\\ (Ask Ubuntu)\end{tabular}}                                                            & \begin{tabular}[c]{@{}l@{}}Máy in\\ (Printer)\end{tabular}                     & 8                                                                        & 12                                                                 & 20       \\ \cline{2-5} 
                                                                                   & \begin{tabular}[c]{@{}l@{}}Phần mềm\\ (Software)\end{tabular}                  & 3                                                                        & 4                                                                  & 7        \\ \cline{2-5} 
                                                                                   & \begin{tabular}[c]{@{}l@{}}Phiên bản \\ (Version)\end{tabular}                 & 24                                                                       & 78                                                                 & 102      \\ \cline{2-5} 
                                                                                   & $\Sigma$                                                                       & 35                                                                       & 94                                                                 & 129      \\ \hline
\end{tabular}
\caption{Các loại thực thể (entity) trong bộ ngữ liệu StackExchange}
    \label{fig:comparisonimg-entityTypesStackExchange}
    \end{center}
\end{table}


\textbf{Phần thí nghiệm}

\begin{itemize}
    \item[--] Tác giả huấn luyện mô hình trên các nền tảng LUIS, Watson Conversation, API.ai, và RASA. Tất cả đều dùng trên cùng một bộ dữ liệu.
        \item[--]Sau đó bộ dữ liệu kiểm thử đã được gửi đến các dịch vụ hiểu ngôn ngữ tự nhiên (\ac{nlu}) để gán nhãn, so sánh gán nhãn dự đoán bởi dịch vụ hiểu ngôn ngữ tự nhiên (\ac{nlu}) với nhãn thực tế.
        \item[--]Để mà đánh giá kết quả, tác giả đã tính toán các độ đo precision và recall cũng như F-score cho từng ý định (intent), thực thể (entity), từng bộ dữ liệu, cũng như kết quả tổng quát. Một dịch vụ tốt hơn dịch vụ kia nếu F-score của nó cao hơn.
\end{itemize}

\textbf{Đánh giá}

Kết quả thu được được mô tả dưới hình sau \ref{fig:comparisonimg-FscoresNLUServices}

\begin{figure}[H]
    \centering
    \includegraphics[width=15cm]{images/comparisonimg/FscoresNLUServices.png}
    \caption{F1-score của các dịch vụ hiểu ngôn ngữ tự nhiên (\ac{nlu}) theo bộ ngữ liệu}
    \label{fig:comparisonimg-FscoresNLUServices}
\end{figure}

\begin{itemize}
    \item[--] Tổng quát LUIS có hiệu năng tốt nhất với F-score là 0.916, tiếp sau đó là là Rasa (0.821), Watson Conversation (0.752), and API.ai (0.687). Trên từng bộ dữ liệu: chatbot, web apps và ask ubuntu, LUIS cũng có hiệu năng cao nhất. Tương tự , API.ai có hiệu năng thấp nhất trên từng bộ dữ liệu, trong khi đó vị trí thứ 2 thay đổi giữa Watson và Rasa.
    \item[--] Dựa trên dữ liệu này, hiệu suất tốt nhất thuộc về sản phẩm thương mại LUIS, tuy nhiên phần mềm nguồn mở Rasa hiệu năng cao hơn các sản phẩm thương mại còn lại.
\end{itemize}

\textbf{II. Đo điểm chuẩn các dịch vụ hiểu ngôn ngữ tự nhiên cho việc xây dựng hệ thống hội thoại.}

Tương tự như bài báo trên, bài này cũng so sánh hiệu năng giữa các dịch vụ hiểu ngôn ngữ tự nhiên (\ac{nlu}) bao gồm các sản phẩm thương mại: Dialogflow\footnote{\url{https://cloud.google.com/dialogflow/docs/}}, LUIS\footnote{\url{https://www.luis.ai/}} và Watson\footnote{\url{https://www.ibm.com/watson}}, mã nguồn mở gồm Rasa\footnote{\url{https://rasa.com/}}.


\textbf{Dữ liệu: }

\begin{itemize}
    \item[--] Dữ liệu được thu từ người dùng thật thông qua Amazon Mechanical Turk (AMT). Các câu hỏi thuộc về cách mọi người tương tác với robot gia đình trong các tình huống được thiết kế trước, cụ thể là: báo thức (alarm), âm thanh (audio), sách nói (audiobook), lịch (calendar), nấu nướng (cooking), ngày giờ (datetime), thư điện tử (email), trò chơi (game), tổng hợp (general), IoT, danh sách "lists", âm nhạc (music), tin tức (news), podcasts, general QA, đài (radio), khuyến nghị (recommendations), xã hội (social), thức ăn mua về (food takeaway), vận chuyển (transport), và thời tiết (weather).
    \item[--] Sau khi thu thập, gán nhãn và kiểm tra, dữ liệu cuối cùng thu được bao gồm 25716 câu nói, gồm 64 ý định (intent) và 54 loại thực thể (entity).
\end{itemize}

\textbf{Thí nghiệm đánh giá}

Dữ liệu huấn luyện và kiểm thử
\begin{itemize}
    \item[--] Vì LUIS giới hạn kích thước dữ liệu huấn luyện, nên mỗi ý định (intent) chọn ra ngẫu nhiên 190 câu nói. Một số ý định (intent) có ít hơn 190 câu nói một chút.
    \item[--] Dữ liệu thu được là 11036 câu nói thuộc 64 ý định (intent) và 54 loại thực thể (entity).
    \item[--] Đối với thử nghiệm đánh giá, tác giả sử dụng 10 fold cross-validation \footnote{\url{https://en.wikipedia.org/wiki/Cross-validation_(statistics)}} với 90\% dữ liệu huấn luyện và 10\% dữ liệu kiểm thử.
\end{itemize}

\textbf{Kết quả}

Dữ liệu từ bảng \ref{fig:overallScores4IntentandEntity} cho thấy điểm trung bình cho phân lớp ý định (intent) và thực thể (entity) thông qua 10-fold cross validation.

Dữ liệu bảng \ref{fig:combinedOverallScores} cho thấy điểm F-score trung bình cho mỗi nền tảng.

\begin{table}[]
 \begin{center}
\begin{tabular}{|l|l|l|l|l|l|l|}
\hline
\multirow{2}{*}{\textbf{}} & \multicolumn{3}{l|}{\textbf{\begin{tabular}[c]{@{}l@{}}Ý định\\ (Intent)\end{tabular}}} & \multicolumn{3}{l|}{\textbf{\begin{tabular}[c]{@{}l@{}}Thực thể\\ (Entity)\end{tabular}}} \\ \cline{2-7} 
                           & \textbf{Prec}          & \textbf{Rec}         & \textbf{F1}                             & \textbf{Prec}          & \textbf{Rec}          & \textbf{F1}                              \\ \hline
Rasa                       & 0.863                  & 0.863                & 0.863                                   & 0.859                  & 0.694                 & 0.768                                    \\ \hline
Dialogflow                 & 0.870                  & 0.859                & 0.864                                   & 0.782                  & 0.709                 & 0.743                                    \\ \hline
LUIS                       & 0.855                  & 0.855                & 0.855                                   & 0.837                  & 0.725                 & 0.777          \\ \hline
Watson                     & 0.884                  & 0.881                & 0.882        & 0.354                  & 0.787                 & 0.488         \\ \hline
\end{tabular}
 \caption{Các chỉ số của nhận diện ý định (intent) và thực thể (entity)}
    \label{fig:overallScores4IntentandEntity}
    \end{center}
\end{table}


\begin{table}[]
\begin{center}
\begin{tabular}{|l|l|l|l|}
\hline
\textbf{}  & Prec  & Rec   & F1             \\ \hline
Rasa       & 0.862 & 0.787 & 0.822          \\ \hline
Dialogflow & 0.832 & 0.791 & 0.811          \\ \hline
LUIS       & 0.848 & 0.796 & 0.821          \\ \hline
Watson     & 0.540 & 0.838 & 0.657 \\ \hline
\end{tabular}
    \caption{Điểm trung bình của cả nhận diện ý định (intent) và thực thể (entity)}
    \label{fig:combinedOverallScores}
    \end{center}
\end{table}

\begin{itemize}
    \item[--] Dựa vào bảng \ref{fig:overallScores4IntentandEntity}, đối với ý định (intent), không có sự khác biệt lớn nào giữa Dialogflow, LUIS và Rasa. F1-score của Watson (0.882) tương đối cao hơn so với các nền tảng khác. Tuy nhiên đối với nhận diện thực thể (entity), Watson có F1-score thấp do Precision rất thấp.
    \item[--] Bảng \ref{fig:combinedOverallScores} chỉ ra rằng tất cả các dịch vụ có F1-score tương đương nhau ngoại trừ Watson có điểm thấp hơn đáng kể.
    \item[--] Trong bài so sánh này, ta thấy hiệu năng của Watson thấp nhất với F1-score là 0.657. Những nền tảng còn lại là Dialogflow, LUIS và RASA có hiệu năng tương đối cao, trong đó nền tảng mã nguồn mở Rasa có hiệu năng tốt nhất (0.822)
\end{itemize}

\textbf{III. Nền tảng giọng nói Snips: một hệ thống hiểu ngôn ngữ tự nhiên nhúng cho các giao diện giọng nói riêng tư theo thiết kế.}

\begin{itemize}
    \item[--] Thử nghiệm này được thực hiện giống với phương pháp của bài báo đã được xuất bản trước đó. Evaluating Natural Language Understanding Services for Conversational Question Answering Systems \cite{EvaluatingNLU}. So sánh hiệu năng giữa các dịch vụ hiểu ngôn ngữ tự nhiên (\ac{nlu}) khác nhau: một số giải pháp dựa trên đám mây phổ biến (Microsoft’s Luis, IBM Watson, API.AI nay là Dialogflow của Google), nền tảng mã nguồn mở Rasa NLU và Snips NLU.
    \item[--] Chỉ số chính được sử dụng trong điểm chuẩn này là F1-score trung bình của phân loại ý định (intent) và các thực thể (entity). Xem kết quả thô\footnote{\url{https://github.com/sonos/nlu-benchmark}}. 
\end{itemize}

\textbf{Dữ liệu}
\begin{itemize}
    \item[--] Dữ liệu bao gồm ba kho tài liệu. Hai trong đó được trích xuất từ StackExchange, một từ chatbot Telegram. Hình \ref{fig:averageF1-scores} trình bày trung bình kết quả trên ba kho tài liệu. Kết quả được trình bày trong bảng \ref{fig:benchmark2017and2018}. Rasa được xem xét trên cả ba phần backend có thể có (Spacy, SKLearn + MITIE, MITIE). Tuy nhiên vì lý do thời gian huấn luyện, chỉ có Spacy được chạy trên cả 3 bộ dữ liệu.
    \item[--] Để công bằng, phiên bản mới nhất của Rasa NLU cũng được sử dụng.

\end{itemize}


\begin{table}[]
\begin{center}
\begin{tabular}{|l|l|l|l|l|}
\hline
\begin{tabular}[c]{@{}l@{}}Bộ dữ liệu\\ (Corpus)\end{tabular} & \begin{tabular}[c]{@{}l@{}}Nhà cung cấp NLU\\ (NLU provider)\end{tabular} & Precision & Recall & F1-score \\ \hline
Chatbot    & \begin{tabular}[c]{@{}l@{}}Luis*\\ IBM Watson*\\ API.ai*\\ Rasa*\\ Rasa**\\ Snips**\end{tabular} & \begin{tabular}[c]{@{}l@{}}0.970\\ 0.686\\ 0.936\\ 0.970\\ 0.933\\ 0.963\end{tabular} & \begin{tabular}[c]{@{}l@{}}0.918\\ 0.8\\ 0.532\\ 0.918\\ 0.921\\ 0.899\end{tabular}   & \begin{tabular}[c]{@{}l@{}}0.943\\ 0.739\\ 0.678\\ 0.943\\ 0.927\\ 0.930\end{tabular} \\ \hline
\begin{tabular}[c]{@{}l@{}}Ứng dụng web\\ (Web apps)\end{tabular}   & \begin{tabular}[c]{@{}l@{}}Luis*\\ IBM Watson*\\ API.ai*\\ Rasa*\\ Rasa**\\ Snips**\end{tabular} & \begin{tabular}[c]{@{}l@{}}0.828\\ 0.828\\ 0.810\\ 0.466\\ 0.593\\ 0.655\end{tabular}          & \begin{tabular}[c]{@{}l@{}}0.653\\ 0.585\\ 0.382\\ 0.724\\ 0.613\\ 0.655\end{tabular} & \begin{tabular}[c]{@{}l@{}}0.73\\ 0.686\\ 0.519\\ 0.567\\ 0.603\\ 0.655\end{tabular}  \\ \hline
\begin{tabular}[c]{@{}l@{}}Hỏi Ubuntu\\ (Ask Ubuntu)\end{tabular} & \begin{tabular}[c]{@{}l@{}}Luis*\\ IBM Watson*\\ API.ai*\\ Rasa*\\ Rasa**\\ Snips**\end{tabular} & \begin{tabular}[c]{@{}l@{}}0.885\\ 0.807\\ 0.815\\ 0.791\\ 0.796\\ 0.812\end{tabular}          & \begin{tabular}[c]{@{}l@{}}0.842\\ 0.825\\ 0.754\\ 0.823\\ 0.768\\ 0.828\end{tabular} & \begin{tabular}[c]{@{}l@{}}0.863\\ 0.816\\ 0.783\\ 0.807\\ 0.782\\ 0.820\end{tabular} \\ \hline
\begin{tabular}[c]{@{}l@{}}Tổng thể\\ (Overall)\end{tabular}    & \begin{tabular}[c]{@{}l@{}}Luis*\\ IBM Watson*\\ API.ai*\\ Rasa*\\ Rasa**\\ Snips**\end{tabular} & \begin{tabular}[c]{@{}l@{}}0.945\\ 0.738\\ 0.871\\ 0.789\\ 0.866\\ 0.896\end{tabular}          & \begin{tabular}[c]{@{}l@{}}0.889\\ 0.767\\ 0.567\\ 0.855\\ 0.856\\ 0.858\end{tabular} & \begin{tabular}[c]{@{}l@{}}0.916\\ 0.752\\ 0.687\\ 0.821\\ 0.861\\ 0.877\end{tabular} \\ \hline
\end{tabular}
    \caption{Precision, recall và F1-score trên 3 bộ dữ liệu}
    \label{fig:benchmark2017and2018}
    \end{center}
\end{table}


\begin{figure}[H]
    \centering
    \includegraphics[width=15cm]{images/comparisonimg/averageF1-scores.png}
    \caption{Trung bình F1-score của phân loại ý định (intent) và thực thể (entity) trên 3 bộ dữ liệu}
    \label{fig:averageF1-scores}
\end{figure}

\textbf{Kết quả}
\begin{itemize}
    \item[--] Dựa trên dữ liệu bảng \ref{fig:benchmark2017and2018}, hiệu suất tốt nhất thuộc về sản phẩm thương mại LUIS (F1-score 0.916), trong khi đó cả vị trí thứ 2 và thứ 3 đều thuộc về sản phẩm nguồn mở Snips (F1-score 0.877) và Rasa  (F1-score 0.861).
    \item[--] Trong bài này nhóm em quan tâm đến kết quả so sánh giữa Rasa với Snips (đều là 2 nền tảng nguồn mở). Khi cả 2 đều sử dụng phiên bản mới nhất và được đánh giá trên cùng một bộ dữ liệu, cả Rasa và Snips đều thể hiện hiệu năng tốt và không chênh lệch nhau nhiều. Tuy nhiên F1-score của Snips nhỉnh hơn một chút so với Rasa (F1-score của Snips là 0.877 và của Rasa là 0.861)
\end{itemize}

\subsection{Kết luận và lựa chọn nền tảng sử dụng}

\begin{itemize}
    \item[--] Dựa trên kết quả các bài báo mà nhóm tham khảo. Nhóm em rút ra được rằng: Các nền tảng nguồn mở vẫn có thể cạnh tranh được với các nền tảng thương mại. Mặc dù không có hiệu năng tốt nhất trên bảng so sánh, nhưng cả 2 nền tảng nguồn mở (Rasa và Snips) đều đứng ở vị trí thứ 2 và thứ 3 trong số 5 nền tảng được so sánh.
    \item[--] Dựa trên các tiêu chí lựa chọn, nhóm em xem xét 2 nền tảng mã nguồn mở đó là Rasa và Snips để sử dụng trong nghiên cứu này. 
    \item[--] Nhóm quyết định chọn Snips NLU do nó có hiệu năng tốt hơn, được cài đặt và sử dụng đơn giản, có nhiều tài liệu hướng dẫn và khả năng tùy chỉnh dễ dàng.
\end{itemize}