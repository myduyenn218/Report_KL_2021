\chapter*{Tóm tắt}
\label{tomtat}
Hiện nay, các chatbot trên thế giới đã phát triển và có khả năng chat gần như con người. Tuy nhiên, điều này chỉ đúng với các chatbot sử dụng ngôn ngữ khác tiếng Việt. Vì vậy, nhóm chúng em muốn thực hiện đề tài xây dựng giải pháp trả lời tự động sử dụng ngôn ngữ tiếng Việt, giao tiếp bằng giọng nói và văn bản với con người. Để thực hiện mục tiêu này, nhóm chúng em đã nghiên cứu về cách xây dựng chatbot với \ac{nlu}, sau đó sử dụng các thuật toán và \ac{api} có sẵn để phân tích, trích xuất nội dung để có thể hiểu câu hỏi và tìm kiếm câu trả lời phù hợp. Nhóm đã quyết định xây dựng một ứng dụng chatbot trên thiết bị di động, với chủ đề là các vấn đề về hướng dẫn đường đi, khoảng cách, vị trí,...

Sau sáu tháng nghiên cứu và xây dựng, nhóm chúng em đã căn bản xây dựng thành công hệ thống này bao gồm ứng dụng chatbot trên thiết bị di động với dữ liệu đầu vào và kết quả đầu ra khi chat là cả giọng nói (audio) và văn bản (text) với ngôn ngữ sử dụng là tiếng Việt, tạo một bộ dữ liệu huấn luyện chatbot. Trong đó, sử dụng thư viện SpeechToText và TTSpeech để chuyển đổi giọng nói thành văn bản và ngược lại. Nghiên cứu bộ thư viện Snips NLU - mã nguồn mở để phân tích ý định và trích xuất nội dung. Từ đó, sử dụng Google Map API để tìm kiếm các phản hồi phù hợp (Search compatible response) về các vấn đề như hỏi đường, hỏi địa chỉ, hỏi khoảng cách,... Tạo một bộ dữ liệu dùng để huấn luyện bằng tiếng Việt và thực hiện dịch (translate) từ tiếng Việt sang tiếng Anh bằng bộ từ điển do chúng em tự xây dựng kết hợp với sử dụng thư viện VNCoreNLU để tách và gán nhãn từ để kết quả dịch được chính xác hơn.

Tuy nhiên, sản phẩm vẫn còn điểm yếu là độ chính xác chưa cao (90\%), bộ dữ liệu huấn luyện còn đơn giản (3200 câu huấn luyện và 120 câu kiểm thử) với 4 loại ý định thuộc lĩnh vực hỏi đường đi trong phạm vi Thành phố Thủ Đức. Nhóm chúng em sẽ tiếp tục nghiên cứu và hoàn thiện hơn sản phẩm trong thời gian sắp tới. 
